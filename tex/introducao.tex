%--------------------------------------------------------------------------------------
% Este arquivo contém a sua introdução, objetivos e organização do trabalho
%--------------------------------------------------------------------------------------
\chapter{Introdução}
PRECISA INSERIR O TEXTO QUE COLOQUEI NO DRIVE
\section{Justificativa}

Diversas iniciativas reforçam o crescimento da Educação a Distância (EAD) e
exigem uma atenção maior nos aspectos importantes para a consolidação e
manutenção das atividades dessa modalidade nas instituições. Dentre esses
aspectos, está a necessidade de pesquisas voltadas para a modalidade, como forma
de se agregar procedimentos validados cientificamente, ferramentas de gestão e
avaliação mais eficientes e abrangentes e metodologias inovadoras e capazes de
superar grandes desafios impostos pela EAD, tais como a adequação das
metodologias de ensino, a gestão acadêmica e as altas taxas de evasão
verificadas na modalidade.

O avanço da modalidade de EAD requer estudos que apontem maneiras cada vez mais
consistentes de tornar a modalidade eficiente. Projetar recursos que suportem as
atividades de acompanhamento em cursos oferecidos em um ambiente virtual de
ensino poderá contribuir significativamente para este propósito. Os recursos a
serem propostos podem ser obtidos a partir de metodologias de análise que
envolvem o conhecimento das estratégias pedagógicas dos ambientes virtuais
atualmente em uso, o levantamento das necessidades apontadas pelos profissionais
que atuam na área e na elicitação dos requisitos para implementação de
ferramentas de visualização de dados de diversas atividades dentro de um curso
on-line.

Com a expansão do EAD de maneira responsável e planejada, com infraestrutura
compatível e recursos humanos qualificados, será possível a oferta de novos
cursos pelas instituições, disseminando conhecimento e possibilitando mais
oportunidades para o desenvolvimento regional.

Os altos índices de evasão dos alunos em cursos na educação a distância
representa um grande desafio para todos os que atuam na modalidade. Além desses
índices estarem em níveis elevados, observou-se também que estão crescendo, o
que demanda por pesquisas e ações que busquem não somente estancar esse
crescimento, mas também reduzir essas taxas. Há uma necessidade contínua de
desenvolvimento de pesquisas que apontem caminhos, métodos e ferramentas que os
auxiliem a enfrentar melhor esse problema. O uso de técnicas estatísticas e de
mineração de dados, em conjunto com teorias consolidadas na modalidade, pode
fornecer mecanismos eficientes de detecção precoce do risco de evasão pelos
alunos (REF).

Espera-se com isso contribuir para o fortalecimento da modalidade, além de
fomentar a linha de pesquisa voltada para o estudo das tecnologias educacionais,
tão evidenciadas e diversificadas a partir do uso cada vez maior das tecnologias
de informação e comunicação no processo de ensino/aprendizagem.

\section{Objetivos gerais}

\lipsum[1]

\section{Objetivos específicos}

\begin{itemize}
	\item blablablabla;
    \item blablablabla;
    \item blablablabla.
\end{itemize}

\section{Organização do trabalho}

\lipsum[10-12]
