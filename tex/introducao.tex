%--------------------------------------------------------------------------------------
% Este arquivo contém a sua introdução, objetivos e organização do trabalho
%--------------------------------------------------------------------------------------
\chapter{Introdução}

O desafio de levar educação e formação profissional a lugares remotos, onde,
dificilmente, a formação presencial tradicional conseguiria alcançar de maneira
efetiva, é uma das principais bandeiras da Educação a Distância (EAD).
Entretanto, outros desafios surgem em decorrência da expansão da modalidade:
como garantir a qualidade dessa formação? como transpor modelos de educação
presencial para a distância? como atuar de maneira a prevenir e reduzir os altos
índices de evasão, ainda, verificados na modalidade? São questões como essas que
devem ser respondidas a partir do desenvolvimento de pesquisas nessa modalidade.
O uso de novas tecnologias e de novos processos pode contribuir com essas
pesquisas.

A partir do uso de algoritmos de aprendizagem de máquina já consolidados na área
de mineração de dados, pretende-se testar modelos preditivos da evasão de alunos
na EAD, que foram testados em alunos e cursos de outra instituição, para
avaliar a sua aplicabilidade nos cursos de graduação a distância
ofertados pela Universidade Federal do Vale do São Francisco (UNIVASF).

\section{Justificativa}

Diversas iniciativas reforçam o crescimento da EAD e exigem uma atenção maior
nos aspectos importantes para a consolidação e manutenção das atividades dessa
modalidade nas instituições. Dentre esses aspectos, está a necessidade de
pesquisas voltadas para a modalidade, como forma de se agregar procedimentos
validados cientificamente, ferramentas de gestão mais eficientes e metodologias
inovadoras, capazes de superar grandes desafios impostos pela EAD.

O avanço da modalidade de EAD requer o desenvolvimento de recursos que permitam
o acompanhamento de cursos oferecidos em um ambiente virtual de aprendizagem.
Esses recursos podem ser obtidos a partir de metodologias de análise que
envolvem o conhecimento das estratégias pedagógicas dos cursos EAD, o
levantamento das necessidades apontadas pelos profissionais que atuam na área e
na elicitação dos requisitos para implementação de ferramentas de visualização
de dados de diversas atividades dentro de um contexto educacional.
\cite{ramos2016abordagem}

Com a expansão do EAD de maneira responsável e planejada, com infraestrutura
compatível e recursos humanos qualificados, será possível a oferta de novos
cursos pelas instituições, disseminando conhecimento e possibilitando mais
oportunidades para o desenvolvimento regional.

Os altos índices de evasão dos alunos em cursos EAD representam um grande
desafio para todos os que atuam na modalidade. Além desses índices estarem em
níveis elevados, observou-se também que estão em crescimento. Com isso há uma
necessidade contínua de desenvolvimento de pesquisas que apontem caminhos,
métodos e ferramentas que os auxiliem a enfrentar melhor esse problema. O uso de
técnicas estatísticas e de mineração de dados, em conjunto com teorias
consolidadas na modalidade, pode fundamentar modelos eficientes de detecção
precoce do risco de evasão pelos alunos \cite{ramos2016abordagem}.

Espera-se com este trabalho contribuir para o fortalecimento da EAD, além de
fomentar a linha de pesquisa voltada para o estudo das tecnologias educacionais,
tão evidenciadas e diversificadas, a partir do uso cada vez maior das
tecnologias de informação e comunicação no processo de ensino/aprendizagem.

É possível utilizar mineração de dados educacionais para prever a taxa de evasão
de alunos em cursos EAD, utilizando modelos construídos a partir de variáveis
selecionadas com base nos construtos da TDT?

\section{Objetivos}

Esta pesquisa será desenvolvida com o propósito de atingir os seguintes
objetivos geral e específicos:

\subsection{Objetivo geral}

Avaliar se um conjunto de modelos de predição desenvolvidos para outro cenário
de EAD, pode também ser usado para prever quais alunos têm tendência a evasão em
cursos nessa modalidade na UNIVASF, mantendo os resultados em níveis
satisfatórios, comparados aos originais.

\subsection{Objetivos específicos}
\begin{itemize}
  \item Adaptar os modelos preditivos já desenvolvidos para uma outra ferramenta
  tecnológica;
  \item Aplicar os classificadores em bases de dados de cursos EAD da UNIVASF;
  \item Avaliar os resultados dos classificadores segundo métricas consolidadas.
\end{itemize}
