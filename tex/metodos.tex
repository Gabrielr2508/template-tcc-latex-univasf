%--------------------------------------------------------------------------------------
% Este arquivo contém a sua metodologia
%--------------------------------------------------------------------------------------
\chapter{Materiais e Métodos} \label{ch:MM} %Uma label é como você referencia uma seção no texto com a tag \ref{}


\section{Base de dados} 
Os dados utilizados para este trabalho consistem em 9414 coletas de MEG para treino, realizadas em 16 indivíduos diferentes, com a indicação de qual estimulo o indivíduo em questão foi exposto, face ou face embaralhada. E também 4058 coletas para teste, realizadas em 7 indivíduos, sem as respectivas classes de estimulo descriminadas. Cada sujeito tem aproximadamente entre 580 a 590 coletas disponíveis para ambos os grupos. 

A coleta é constituída de 1.5 segundos de gravação do MEG, começando 0.5 segundos antes do estimulo começar. Para os dados de treino, inclui-se também a classe do respectivo estimulo, face (1) ou face embaralhada (0). Originalmente os dados foram coletados numa frequência de 1 Khz, contudo os dados disponibilizados foram sub-amostrados para frequência  de 250 hz e um filtro passa alta foi aplicado em 1hz para eliminar componentes contínuas (CC) que não compõe o sinal de interesse.

Os procedimentos descritos anteriormente foram realizados com a biblioteca nme-python e os dados foram arranjados em uma matriz tridimensional que indexa Coleta x Canal x Tempo, com tamanho de  580 - 590 x 306 x 375. Para o conjunto de treino, esta matriz está associada a um vetor cuja dimensão é igual à primeira dimensão da matriz, no qual estão às \textit{labels} associadas.

Os sensores utilizados foram espalhados como demonstrado na (Imagem), agrupados em grupos de três sensores por local, cobrindo 102 locais. Em cada local há dois gradiômetros ortogonais e um magnetômetro que fazem a gravação do campo magnético induzido pelas ondas cerebrais. Os gradiômetros medem a variação espacial do campo magnético, dando origem ao plano XY e o magnetômetro mede a componente radial Z do mesmo.   


%\subsection{Subseção de exemplo 1 - Referenciando seções} \label{subsec:subsec1}






%--------------------------------------------------------------------------------------
% Insere a seção de cronograma
% Está comentada porque só é necessária no TCC I
%--------------------------------------------------------------------------------------

%\section{Cronograma} \label{sec:crono}

%A tabela \ref{tab:cronograma} mostra o cronograma de atividades a serem executadas para o TCC II, com base no calendário de 201X.Y da UNIVASF.

%\newpage
%\begin{table}[!thb]
%	%\huge
%    \centering
%    \caption{\label{tab:cronograma} Cronograma das atividades previstas para o TCC II}
%%    \begin{adjustbox}{max width=\textwidth}
%    \begin{tabular}{p{6.5cm}|c|c|c|c|c|c}
%    \toprule
%    \textbf{Atividade}                      & Nov & Dez & Jan & Fev & Mar & Abr \\ \hline
%    Implementar o banco de dados              & X    & X     &       &        &          &          \\ \hline
%    Desenvolver a API HTTP RESTful                      &   X   & X     &       &        &          &          \\ \hline
%    Implementar o serviço de captura de dados        &      &      & X     &   X     &          &          \\ \hline
%    Desenvolver a aplicação \textit{Web/mobile} para exibição dos dados         &      &      & X     &   X     &     X     &          \\ \hline
 %   Teste do sistema            &      &       &       &        & X        &          %\\ \hline
 %   Escrita do TCC II                       &   X   & X     & X     & X      & X        & X        \\ \hline
%   Defesa do TCC II                        &      &       &       &        &          & X       \\
%    \bottomrule
 %   \end{tabular}
 %   \end{adjustbox}
%    \legend{\textbf{Fonte:} O autor.}
%\end{table}

