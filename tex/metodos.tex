\chapter{Metodologia proposta}

\section{Caracterização da pesquisa}

Segundo \citeonline{marconi2003fundamentos}, a pesquisa é um procedimento
formal, com método de pensamento reflexivo, que requer um tratamento científico
e se constitui no caminho para conhecer a realidade ou para descobrir verdades
parciais. A pesquisa é um procedimento sistemático e crítico,  que permite
descobrir novos fatos, relações ou leis acerca de qualquer campo do
conhecimento.

Uma pesquisa pode ser caracterizada segundo os seguintes critérios
\cite{gil2008metodos}:
\begin{itemize}
  \item Quanto à natureza: básica ou aplicada;
  \item Quanto aos objetivos: exploratória, descritiva ou explicativa;
  \item Quanto à abordagem: qualitativa ou quantitativa;
  \item Quanto aos procedimentos: documental, bibliográfica, experimental,
  levantamento, estudo de caso, entre outros.
\end{itemize}

Este trabalho pode ser classificado como de natureza aplicada, já que será
aplicada uma metodologia de busca de conhecimentos em bancos de dados e métodos
de classificação para prever a evasão de cursos EAD.

Em relação aos objetivos podemos classificar este trabalho como pesquisa
exploratória e descritiva. Tendo como base \citeonline{gil2002elaborar}, a
pesquisa exploratória busca ampliar o conhecimento sobre o problema, procurando
torná-lo mais explícito ou a construção de hipóteses, tendo como objetivo
central o aperfeiçoamento de ideias ou a revelação de intuições. E a pesquisa
descritiva objetiva descrever características de determinado fenômeno ou
população. Este trabalho utiliza uma metodologia de exploração de conhecimento
para tentar prever um comportamento em um conjunto de uma população.

Quanto à abordagem, este trabalho é classificado como quantitativo, em razão da
utilização de abordagens algorítmicas de Mineração de Dados, a partir das quais
serão extraídas as características dos estudantes de EAD e aplicadas modelos de
classificação que farão a devida categorização.

No quesito procedimentos classificamos este trabalho como pesquisa experimental.
De acordo com \citeonline{gil2002elaborar}, a pesquisa experimental consiste em
determinar um objeto de estudo, selecionar as variáveis que seriam capazes de
influenciá-lo, definir as formas de controle e de observação dos efeitos  qua
variável produz no objeto.

No caso deste trabalho, o objeto de estudo é a evasão na EAD da UNIVASF e as
variáveis foram definidas pela TDT.
