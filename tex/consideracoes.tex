\chapter{Considerações finais}

A modalidade EAD ajuda a democratizar o ensino, levando-o às regiões de difícil
acesso aos professores ou dando a oportunidade ao estudante de criar sua própria
rotina de estudos. A evasão desta modalidade de ensino ainda é um grande
problema a ser resolvido, logo, existe a necessidade de pesquisa científica
nesta área.

Com o uso crescente de ferramentas de tecnologia da informação em EAD fica
evidente que o uso de aprendizagem de máquina pode ser utilizado para modelar e
prever os fenômenos que causam a evasão.

O fluxo de descoberta de conhecimento em bases de dados descrito na metodologia
deste trabalho será utilizado como arcabouço para o desenvolvimento dos modelos
preditivos que serão comparados no decorrer da disciplina Trabalho de Conclusão
de Curso II (TCC II). Espera-se obter resultados satisfatórios, aproximados aos
encontrados na literatura.
