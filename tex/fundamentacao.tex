%-------------------------------------------------------------------------------
% Este arquivo contém a sua funtamentação teórica
%-------------------------------------------------------------------------------
\chapter{Fundamentação Teórica}

A necessidade de apoiar o desenvolvimento da EAD tem feito surgir novas teorias,
métodos, abordagens de ensino e tratamento das informações, geradas nas diversas
tecnologias usadas na EAD. Serão abordadas nas seções seguintes, as principais
temáticas que serão usadas neste estudo, dando-lhe as bases teóricas necessárias
para o seu desenvolvimento e compreensão dos seus resultados futuros.

\section{Teoria da Distância Transacional}

Em 1972, Michael Grahame Moore propôs uma das primeiras teorias para a educação
a distância. Essa teoria seria posteriormente denominada de Teoria da Distância
Transacional (TDT). Ao longo de mais de 40 anos desde de seus fundamentos
iniciais, o próprio autor e outros pesquisadores trataram de atualizá-la,
principalmente em razão da evolução tecnológica e da necessidade de sua
renovação periódica. Os textos originais do autor estão nas suas obras de 1973,
1993 e 2013 \cite{moore1973transational}.

Em seus estudos, \citeonline{moore2013transational} afirmou que, na EAD
não existe apenas uma distância física entre professores e alunos mas também uma
distância psicológica. Esta é uma teoria que descreve as relações
estudante-professor e intra-estudante e suas consequências na qualidade do
ensino à distância. Na TDT, este universo de relações pode ser estudado
baseando-se na construção dos componentes ou construtos elementares deste campo,
sendo eles, a estrutura dos programas e cursos, a interação entre alunos e
professores ou diálogo e a próprio grau de autonomia e independência do
estudante. Uma teoria educacional que afirma que a educação a distância tem a
sua própria identidade e características pedagógicas distintivas. É a primeira,
se não a única, teoria em EAD que pode ser utilizada para testar hipóteses. Como
outras teorias, pode ser usada para calcular uma heurística, utilizada para
tomada de decisões em projetos de cursos da EAD \cite{moore2013transational}.

\citeonline{dewey1960knowing} elaboraram o conceito de transação, que, conforme
foi exposto posteriormente por \citeonline{boyd1980redefining}, denota a
interação entre o ambiente, os indivíduos e os padrões de comportamento numa
dada situação. Uma transação, em educação à distância, é a interação entre
professores e alunos que estão espacialmente separados. Como foi definido por
\citeonline{moore2013transational}, essa separação cria padrões especiais de
comportamento que afetam tanto o ensino quanto o aprendizado. Derivado da
separação, surge um espaço psicológico e comunicacional propício a
mal-entendidos nas intervenções instrutor-aluno. A esta separação é dado o nome
de Distância Transacional (DT).

Distância Transacional é o espaço psicológico e comunicacional gerado pela
separação entre alunos e professores em um ambiente de educação à distância. O
termo distância se refere a este espaço psicológico e não a distância física
entre os participantes de um curso EAD \cite{goel2012transactional}. Esta
distância é uma função de três construtos: diálogo, estrutura do curso e
autonomia do aluno.

Faz-se necessário lembrar que, segundo \citeonline{moore2013transational} a
distância transacional não é um valor fixo ou dicotômico, na verdade é um valor
relativo e contínuo. E é diferente para cada estudante, mesmo entre os que
compartilham o mesmo curso. Foi apontado por \citeonline{rumble1986planning} que
existe uma DT mesmo em cursos presenciais. A vista desses argumentos, podemos
dizer que a EAD é um subconjunto da educação e os estudos realizados em EAD
podem auxiliar a teoria e prática da educação tradicional. Porém, em uma
situação classificada como EAD, a distância entre os participantes - professores
e alunos - é grande o suficiente para justificar a investigação de técnicas
próprias de ensino-aprendizagem.

Os procedimentos de ensino se dividem em dois grupos, e acontece também um
terceiro grupo de variáveis que descreve o comportamento dos alunos. A DT é uma
função desses três grupos de variáveis. Essas variáveis não são tecnológicas ou
comunicacionais, na verdade, são variáveis em ensino e aprendizagem e na
interação entre ensino e aprendizagem. Em TDT, estes grupos de variáveis recebem
o nome de Diálogo, Estrutura e Autonomia do Aluno \cite{moore2013transational}.

\subsection{Diálogo}

Diálogo foi originalmente definido por \citeonline{moore2013transational} como
sendo interações focadas, positivas, propositais entre o professor e os alunos.
Ainda segundo Moore, o diálogo ocorre entre professores e alunos quando alguém
ensina e os demais reagem. Interações negativas ou neutras não são classificadas
como diálogo. O diálogo deve ser direcionado para o aperfeiçoamento da
compreensão por parte do aluno.

``A extensão e natureza do diálogo são determinadas pela filosofia educacional
da instituição responsável pelo projeto do curso, pelas personalidades do
professor e do aluno, pelo tema do curso e por fatores ambientais.'' \cite[p.
~438]{cabau2018teoria}.

\citeonline{moore2013transational} cita meios de comunicação como um importante
fator ambiental na EAD, no entanto, relata ser importante que outras variáveis
sejam atendidas à medida que a EAD amadurece, as variáveis destacadas por Moore
foram: projeto de curso, seleção e treinamento de instrutores e o estilo de
aprendizagem dos alunos.

É o mediador central da DT e referenciado como medida de aprendizado ao passo
que a DT seria uma medida de não-aprendizado. No entanto, já que o diálogo não
se limita apenas à interação professor-aluno, especialmente com os avanços da
EAD provendo novas formas de interações entre estudantes, diversos pesquisadores
vêm propondo a inclusão de interações/relacionamentos entre alunos no conceito
de diálogo \cite{benson2009addressing,chen1999dimensions,huang2016understanding}