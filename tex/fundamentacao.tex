%-------------------------------------------------------------------------------
% Este arquivo contém a sua funtamentação teórica
%-------------------------------------------------------------------------------
\chapter{Fundamentação Teórica}

A necessidade de apoiar o desenvolvimento da EAD tem feito surgir novas teorias,
métodos, abordagens de ensino e tratamento das informações, geradas nas diversas
tecnologias usadas na EAD. Serão abordadas nas seções seguintes, as principais
temáticas que serão usadas neste estudo, dando-lhe as bases teóricas necessárias
para o seu desenvolvimento e compreensão dos seus resultados futuros.

\section{Teoria da Distância Transacional}

Em 1972, Michael Grahame Moore propôs uma das primeiras teorias para a educação
a distância. Essa teoria seria posteriormente denominada de Teoria da Distância
Transacional (TDT). Ao longo de mais de 40 anos desde de seus fundamentos
iniciais, o próprio autor e outros pesquisadores trataram de atualizá-la,
principalmente em razão da evolução tecnológica e da necessidade de sua
renovação periódica. Os textos originais do autor estão nas suas obras de 1973,
1993 e 2013 \cite{moore1973transational}.

Em seus estudos, \citeonline{moore2008teoria} afirmou que, na EAD
não existe apenas uma distância física entre professores e alunos mas também uma
distância psicológica. Esta é uma teoria que descreve as relações
estudante-professor e intra-estudante e suas consequências na qualidade do
ensino à distância. Na TDT, este universo de relações pode ser estudado
baseando-se na construção dos componentes ou construtos elementares deste campo,
sendo eles, a estrutura dos programas e cursos, a interação entre alunos e
professores ou diálogo e a próprio grau de autonomia e independência do
estudante. Uma teoria educacional que afirma que a educação a distância tem a
sua própria identidade e características pedagógicas distintivas. É a primeira,
se não a única, teoria em EAD que pode ser utilizada para testar hipóteses. Como
outras teorias, pode ser usada para calcular uma heurística, utilizada para
tomada de decisões em projetos de cursos da EAD \cite{moore2008teoria}.

\citeonline{dewey1960knowing} elaboraram o conceito de transação, que, conforme
foi exposto posteriormente por \citeonline{boyd1980redefining}, denota a
interação entre o ambiente, os indivíduos e os padrões de comportamento numa
dada situação. Uma transação, em educação à distância, é a interação entre
professores e alunos que estão espacialmente separados. Como foi definido por
\citeonline{moore2008teoria}, essa separação cria padrões especiais de
comportamento que afetam tanto o ensino quanto o aprendizado. Derivado da
separação, surge um espaço psicológico e comunicacional propício a
mal-entendidos nas intervenções instrutor-aluno. A esta separação é dado o nome
de Distância Transacional (DT).

Distância Transacional é o espaço psicológico e comunicacional gerado pela
separação entre alunos e professores em um ambiente de educação à distância. O
termo distância se refere a este espaço psicológico e não a distância física
entre os participantes de um curso EAD \cite{goel2012transactional}. Esta
distância é uma função de três construtos: diálogo, estrutura do curso e
autonomia do aluno.

Faz-se necessário lembrar que, segundo \citeonline{moore2008teoria} a
distância transacional não é um valor fixo ou dicotômico, na verdade é um valor
relativo e contínuo. E é diferente para cada estudante, mesmo entre os que
compartilham o mesmo curso. Foi apontado por \citeonline{rumble1986planning} que
existe uma DT mesmo em cursos presenciais. A vista desses argumentos, podemos
dizer que a EAD é um subconjunto da educação e os estudos realizados em EAD
podem auxiliar a teoria e prática da educação tradicional. Porém, em uma
situação classificada como EAD, a distância entre os participantes - professores
e alunos - é grande o suficiente para justificar a investigação de técnicas
próprias de ensino-aprendizagem.

Os procedimentos de ensino se dividem em dois grupos, e acontece também um
terceiro grupo de variáveis que descreve o comportamento dos alunos. A DT é uma
função desses três grupos de variáveis. Essas variáveis não são tecnológicas ou
comunicacionais, na verdade, são variáveis em ensino e aprendizagem e na
interação entre ensino e aprendizagem. Em TDT, estes grupos de variáveis recebem
o nome de Diálogo, Estrutura e Autonomia do Aluno \cite{moore2008teoria}.

\subsection{Diálogo}

Diálogo foi originalmente definido por \citeonline{moore2008teoria} como
sendo interações focadas, positivas, propositais entre o professor e os alunos.
Ainda segundo Moore, o diálogo ocorre entre professores e alunos quando alguém
ensina e os demais reagem. Interações negativas ou neutras não são classificadas
como diálogo. O diálogo deve ser direcionado para o aperfeiçoamento da
compreensão por parte do aluno.

``A extensão e natureza do diálogo são determinadas pela filosofia educacional
da instituição responsável pelo projeto do curso, pelas personalidades do
professor e do aluno, pelo tema do curso e por fatores ambientais.''
\cite[p.~438]{cabau2018teoria}.

\citeonline{moore2008teoria} cita meios de comunicação como um importante
fator ambiental na EAD, no entanto, relata ser importante que outras variáveis
sejam atendidas à medida que a EAD amadurece, as variáveis destacadas por Moore
foram: projeto de curso, seleção e treinamento de instrutores e o estilo de
aprendizagem dos alunos.

É o mediador central da DT e referenciado como medida de aprendizado ao passo
que a DT seria uma medida de não-aprendizado. No entanto, já que o diálogo não
se limita apenas à interação professor-aluno, especialmente com os avanços da
EAD provendo novas formas de interações entre estudantes, diversos pesquisadores
vêm propondo a inclusão de interações/relacionamentos entre alunos no conceito
de diálogo
\cite{benson2009addressing,chen1999dimensions,huang2016understanding}.

\subsection{Estrutura do curso}

A estrutura do curso diz respeito aos elementos do projeto, bem como, divisão do
curso em unidades, objetivos, estratégias institucionais e métodos de avaliação.
Estrutura transmite a flexibilidade ou rigidez dos elementos do curso. É também
responsável pela facilitação ou não-facilitação do diálogo
\cite{moore2008teoria}.

Bem como o diálogo, a estrutura do curso é uma variável qualitativa, e a
medida da estrutura em um programa EAD é, normalmente, determinada pela natureza
dos meios de comunicação empregados, e também pela filosofia e personalidade dos
professores, pelas personalidades dos alunos e pelas restrições impostas pelas
instituições educacionais \cite{moore2008teoria}.

Embora Moore atribua como qualitativa o tipo de variável relacionada ao diálogo
e à estrutura, diversos estudos recentes mostraram que é possível quantificar e
mensurar esses componentes da TDT
\cite{zhang2003transactional,horzum2011developing,paul2015revisiting,
ramos2016abordagem}.

Em cursos gravados em fitas, discos, ou mesmo cursos televisionados a estrutura
é rígida e o diálogo não existe, pois não é possível reorganizar o conteúdo para
levar em consideração as interações de um aluno. Em contrapartida cursos por
teleconferências, permitem ampla variedade de respostas alternativas do
instrutor às perguntas dos participantes. Um curso altamente estruturado não
possibilita o diálogo professor-aluno, consequentemente a DT entre alunos e
professores aumenta. No entanto, o contrário não pode ser generalizado. ``\ldots
a extensão do diálogo e a flexibilidade da estrutura variam de programa para
programa. É esta variação que dá a um programa maior ou menor distância
transacional que outro'' \cite{moore2008teoria}.

Em um programa com pequena DT os alunos recebem instruções e orientações por
meio do diálogo com o instrutor, neste caso é possível ter uma estrutura aberta,
que dê respaldo para tais interações. Em programas com maior DT é necessário uma
estrutura robusta, materiais didáticos que forneçam todas as orientações,
instruções e aconselhamentos que o instrutor puder prever, mas sem a
possibilidade de alterações por meio de diálogo aluno-professor
\cite{moore2008teoria}.

Temos então que, em programas mais distantes, os alunos precisam se
responsabilizar em escolher quais atividades e avaliações serão feitas e quando
serão feitas. Mesmo que o curso seja bem estruturado, o estudante, na falta de
diálogo, decidirá quais atividades serão realizadas, quando, e qual a
importância de cada uma. Sendo assim, quanto maior a DT mais é exigido uma
autonomia do aluno \cite{moore2008teoria}.


\subsection{Autonomia do aluno}

No período do surgimento da TDT, década de 1970, ela representava a fusão de
duas tradições pedagógicas presentes nos anos 1960 que pareciam contraditórias.
Uma, a tradição humanística, que valorizava o diálogo aberto, não-estruturado e
interpessoal, tanto na educação quanto no aconselhamento. A outra, a tradição
behaviorista que valorizava o projeto sistemático da instrução, baseado em
objetivos comportamentais com o máximo de controle do processo de aprendizagem
por parte do professor. No início dos anos 1970, a EAD era dominada pela
tradição behaviorista. Tanto que, o título do primeiro trabalho sobre a TDT de
\citeonline{moore1972learner} foi: ``A autonomia do aluno --- a segunda dimensão
da aprendizagem independente''. Neste trabalho Moore afirmou que: ``educadores
por correspondência universitários limitavam o potencial do seu método ao
negligenciar a habilidade dos alunos em compartilharem a responsabilidade por
seus próprios processos de aprendizagem'' \cite{moore2008teoria}.

O termo ``autonomia do aluno'' foi escolhido para descrever os padrões de
comportamento de alunos que usavam materiais didáticos e programas de ensino
para atingir seus próprios objetivos, à sua maneira e sob seu próprio controle
\cite{moore2008teoria}.

\apudonline{boyd1966psychological}{moore2008teoria} criou uma descrição para o
comportamento ideal de uma pessoa emocionalmente independente de um instrutor,
uma pessoa que pode abordar assuntos diretamente sem ter um adulto participando
de um conjunto de papéis de mediação entre o aluno e a matéria de estudo.

Autonomia do aluno  se refere a capacidade de se auto-direcionar.
\citeonline{moore2008teoria} definiu o estudante autônomo ideal como ``a
pessoa emocionalmente independente de um professor'' e quem ``tem capacidade de
abordar o assunto estudado diretamente sem a ajuda de um instrutor''. Diferente
da estrutura do curso e do diálogo, é um fator que depende apenas do aluno. Um
aprendiz pouco autônomo pode precisar de um direcionamento maior e uma estrutura
mais rígida \cite{huang2016understanding}.

\section{Evasão de alunos na EAD}

O censo mais recente, no período da escrita deste trabalho, realizado pela
Associação Brasileira de Educação a Distância (ABED), juntado dados de 2016,
consultou 340 instituições em todo o país, formadoras e fornecedoras de de
produtos e serviços para EAD \cite{abed2016ead}.

De acordo com o Censo EAD.BR 2016 as taxas de evasão informadas pelos
respondentes recaíram principalmente entre 11\% e 25\%. O censo também revelou
que entre os respondentes, as taxas de evasão dos cursos semipresenciais tem
taxa de evasão menor que cursos totalmente a distância. A Tabela
\ref{tableEvasionTax1} compara os índices dos 4 últimos censos realizados pela
ABED
\cite{abed2013ead,abed2014ead,abed2015ead,abed2016ead}

\begin{table}[!htb]
  \centering
  \caption{\label{tableEvasionTax1} Taxas de evasão ao longo dos anos segundo o censo realizado pela ABED}
  \begin{tabular}{@{}lllll@{}}
    \toprule
    \multicolumn{1}{c}{\multirow{2}{*}{\textbf{\begin{tabular}[c]{@{}c@{}}Taxas de evasão \\ declaradas\end{tabular}}}} & \multicolumn{4}{c}{\textbf{\begin{tabular}[c]{@{}c@{}}Percentuais de instituições\\ declarantes, por faixa\end{tabular}}} \\ \cmidrule(l){2-5}
    \multicolumn{1}{c}{} & \textbf{2013} & \textbf{2014} & \textbf{2015} & \textbf{2016} \\ \midrule
    Até 25\% & 65\% & 50\% & 53\% & 58\% \\
    Entre 26 e 50\% & 24\% & 38\% & 40\% & 19\% \\
    Acima de 50\% & 2\% & 2\% & 7\% & 1\% \\
    Não declararam & 9\% & 10\% & - & 22\% \\ \bottomrule
  \end{tabular}
  \legend{\textbf{Fonte:} \citeonline{abed2013ead,abed2014ead,abed2015ead,abed2016ead}}
\end{table}

Entre os motivos para evasão investigados e declarados no censo, questões
financeiras e falta de tempo foram os citados como os que geram maior evasão.
Houve uma parcela considerável de respondentes que acredita que a evasão não é
um problema em cursos totalmente a distância pois os participantes podem sempre
retornar.

Em cursos livres o motivo mais citado foi a falta de tempo, e também grande
parte dos respondentes acredita que os alunos de cursos livres não corporativos
sempre podem retornar.

O Censo EAD.BR 2016 apontou que cursos presenciais, semipresenciais e
corporativos possuem mecanismos que vão além do conteúdo e da interação online
com professores para manter seus alunos engajados. Já os cursos totalmente a
distância e cursos livres não corporativos dependem apenas da experiência do
aluno com o conteúdo e com seus professores e tutores.

A Tabela \ref{tableEvasionTax2} apresenta dados dos indicadores da evasão
segundo o Mapa do Ensino Superior no Brasil Edições 2015 e 2016, que foram
publicados pelo Sindicato das Empresas Mantenedoras do Ensino Superior (SEMESP)
feito com base nos dados do INEP dos anos 2013 e 2014.

\begin{table}[!htb]
  \centering
  \caption{\label{tableEvasionTax2} Taxas de evasão em cursos superiores presenciais e a distância}
  \begin{tabular}{@{}ccccc@{}}
    \toprule
    \multirow{2}{*}{\textbf{Ano}} & \multicolumn{2}{c}{\textbf{Cursos presenciais}} & \multicolumn{2}{c}{\textbf{Cursos a Distância}} \\ \cmidrule(l){2-5}
    & \textbf{IES públicas} & \textbf{IES privadas} & \textbf{IES públicas} & \textbf{IES privadas} \\ \midrule
    2013 & 65\% & 50\% & 53\% & 58\% \\
    2014 & 24\% & 38\% & 40\% & 19\% \\ \bottomrule
  \end{tabular}
  \legend{\textbf{Fonte:} \citeonline{semesp2015mapa,semesp2016mapa}}
\end{table}

Segundo o trabalho de \apudonline{ribeiro2005projeto}{paz2017identificando} a
evasão em instituições de ensino superior (IES) é um tema complexo na gestão
universitária no Brasil. E, segundo
\apudonline{costa2012minerando}{paz2017identificando}, um grave problemas das
universidades brasileiras é o aumento das taxas de evasão escolar.

\citeonline{manhaes2011previsao} identificaram que a descoberta precoce de
grupos de estudantes com risco de evasão é condição importante para reduzir tal
problema pois possibilita proporcionar algum tipo de atendimento personalizado
para a situação de cada aluno. Ainda segundo \citeonline{manhaes2011previsao},
os processos de identificação desses grupos à época eram manuais e sujeitos a
falhas e dependiam primordialmente da experiência do docente.

Motivado pelos problemas elencados acima este trabalho se baseia em técnicas de
mineração de dados e aprendizado de máquina (AM) para tentar categorizar,
previamente, o comportamento de alunos da EAD na Univasf e mitigar os problemas
da evasão escolar.

\section{Relação entre a distância transacional e a evasão em cursos a distância}

Pela sua definição, a distância transacional é um dos fatores que pode gerar
maior dificuldade no engajamento significativo e na comunicação do estudante no
ambiente de aprendizagem \cite{goel2012transactional}. Além de
\citeonline{moore2008teoria}, outros autores afirmaram que quanto maior for a
distância transacional, maior a possibilidade de ocorrência de problemas como
atritos, insatisfações e abandono de cursos
\cite{zhang2003transactional,steinman2007educational,horzum2011developing,
mbwesa2014transactional,paul2015revisiting}.

\citeonline{zhang2003transactional}, em sua tese de doutorado, demonstrou a
existência de uma correlação negativa entre a distância transacional e o
envolvimento dos alunos com a sua aprendizagem, assim como com a sensação de
satisfação e a intenção do aluno em persistir no seu curso on-line.

Os elementos da distância transacional são preditores do engajamento e da
satisfação dos alunos em cursos on-line, o que, de certa forma, são elementos
relacionados com a decisão de permanecer ou abandonar o curso, segundo a revisão
de \citeonline{paul2015revisiting} do trabalho de
\citeonline{zhang2003transactional}.

Para \citeonline{steinman2007educational}, as percepções dos alunos de cursos
on-line podem ser negativas se eles experimentam grande distância transacional
com o instrutor e com outros alunos, podendo ainda influenciar sua decisão de
permanecer ou abandonar o curso. Uma vez que a distância transacional afeta a
satisfação e retenção dos alunos, esse conceito é visto como um importante
tópico de discussão sobre evasão em cursos on-line.

A obtenção dos construtos da distância transacional pode refletir uma condição
ou um estado de um curso no tempo de sua execução, permitindo, por exemplo, que
professores e tutores notem um distanciamento exagerado de determinados alunos e
consigam intervir no sentido de prevenir ou reverter situações de evasão de
alunos do curso \cite{horzum2011developing}.

Em sua pesquisa, \citeonline{mbwesa2014transactional}, procurou estabelecer os
construtos da distância transacional como estimadores da satisfação dos alunos
em um curso a distância. De acordo com os resultados do estudo, as distâncias
transacionais verificadas entre aluno-aluno, aluno-professor e aluno-conteúdo
foram preditores importantes da satisfação percebida dos alunos com os cursos
por EAD.

\section{Mineração de dados e Descoberta de Conhecimento}

Segundo \citeonline{costa2012mineraccao} Mineração de Dados (MD, ou do inglês,
Data Mining, DM), pode ser interpretada como uma etapa de um processo mais amplo
denominado como Descoberta de Conhecimento em Bases de Dados (DCBD, ou do
inglês, Knowledge Discovery in Databases, KDD). Em KDD também são identificadas
duas grandes etapas: a de pré-processamento de dados, onde os dados são captados
tratados e organizados, e a de pós-processamento dos resultados obtidos da etapa
de mineração.

De acordo com \citeonline{fayyad1996data} KDD é o processo não trivial de
identificação de padrões, a partir de dados, que sejam válidos, novos,
potencialmente úteis e compreensíveis. Esta é uma definição abrangente, onde KDD
é um processo de geral de descoberta de conhecimento composto pelas três etapas
mencionadas. Os novos dados gerados devem ser novos, compreensíveis e úteis,
sendo assim, devem trazer novos benefícios de fácil compreensão para ajudar na
tomada de decisões.

No entanto \citeonline{klosgen2002knowledge} descrevem DM como sinônimo de KDD,
descrevendo ambos como uma disciplina que objetiva a extração de automática de
padrões interessantes e implícitos de grandes coleções de dados.

\citeonline{cabena1997discovering} definem DM como uma área interdisciplinar,
mobilizando principalmente conhecimentos de análise estatística de dados,
aprendizagem de máquina, reconhecimento de padrões e visualização de dados.

Com o objetivo de descobrir conhecimentos relevantes em KDD, é necessário
estabelecer metas bem definidas. Segundo \citeonline{fayyad1996data}, as metas
são definidas em função do objetivo na utilização da metodologia, sendo dois
tipos básicos de metas: verificação e descoberta. No caso de verificação, o
sistema está limitado a testar hipóteses definidas pelo usuário, enquanto que em
descoberta o sistema encontra novos padrões de forma autônoma. Quando a meta é
do tipo descoberta, em geral, o objetivo está relacionado com as seguintes
tarefas de mineração de dados: predição e descrição.

Tarefas preditivas tem objetivo de predizer o valor de um determinado atributo
baseado nos valores de outros atributos. O atributo a ser predito pode ser
chamado de variável preditiva, dependente ou alvo, já os atributos utilizados na
predição são chamados de variáveis preditoras, independentes ou explicativas.
Sendo generalista, a predição utiliza um conjunto de variáveis para predizer o
valor de outras \cite{fayyad1996data}.

Tarefas descritivas objetivam encontrar padrões --- correlações, tendências,
grupos, trajetórias e anomalias --- que representem os dados
\cite{fayyad1996data}.

Para realizar tarefas de predição e descrição são utilizados alguma das
seguintes tarefas e métodos de mineração de dados: classificação, regressão,
agrupamento, sumarização, modelagem de dependência e identificação de mudanças e
desvios.

Segundo \citeonline{tan2009introduccao}, DM é uma parte integral do KDD, um
processo geral de conversão de dados brutos em informações úteis. Sendo este
composto de uma séries de passos de transformação, do pré-processamentos dos
dados até o pós-processamento dos resultados da mineração de dados. A Figura
\ref{kddTan} ilustra uma visão geral do KDD segundo Tan et al.

\imagem{0.85}{kdd_tan}{\label{kddTan} Processo de descoberta de conhecimento em bases de dados}{\citeonline{tan2009introduccao}}

Ainda segundo \citeonline{tan2009introduccao} os dados de entrada podem estar
armazenados nos mais diversos formatos (tabelas eletrônicas, bases de dados
estruturadas, arquivos simples), e podem estar em um único repositório em
distribuidos por diversas fontes. A etapa de pré-processamento é responsável por
transformar os dados brutos em dados apropriados para as análises seguintes.
Fusão de dados de múltiplas fontes, limpeza para remoção de ruídos e entradas
duplicadas e seleção de registros e características relevantes à DM são passos
importantes realizados na etapa de pré-processamento. Como existem diversas
formas de se coletar e armazenar os dados, o pré-processamento se torna, muitas
vezes, o trecho mais demorado e trabalhoso do KDD.

De acordo com \citeonline{tan2009introduccao} pós-processamento é a etapa do KDD
onde os dados válidos e úteis gerados na etapa de mineração são integrados a
ferramentas de auxílio na tomada de decisões. Um exemplo de pós-processamento é
a visualização de dados, que permite, por meio de gráficos interativos ou não,
auxiliar na interpretação de comportamentos e características dos dados. Também
podem ser utilizadas estatísticas ou testes de hipóteses para eliminar
resultados não legítimos da mineração de dados.

\subsection{Dados educacionais}

Com o crescente desenvolvimento das tecnologias da informação, e a quantidade de
instituições de EAD que se utilizam de Ambientes Virtuais de Aprendizagem (AVA),
ou mesmo outras tecnologias de apoio ao processo de ensino e aprendizagem, está
sendo gerado um imenso volume de dados relacionados com essa modalidade de
ensino. O volume de dados gerado atualmente é muito maior do que um pesquisador
ou analista  é capaz de analisar, tendo em vista limitações de processamento e a
desestruturação dos dados gerados. Estes aspectos apresentam por si os desafios
ligados ao tratamento desses conjuntos de dados
\cite{rigo2014aplicaccoes,costa2012mineraccao}.

Os dados provenientes dos contextos educacionais tem natureza mais diversa que
os dados utilizados tradicionalmente em DM, necessitando, portanto, adaptações e
novas técnicas. Ao passo que, essas naturezas diversas representam um grande
potencial de implementação de recursos fundamentais para auxílio na melhoria da
educação, por exemplo, a criação de alertas, apoio a sistemas de recomendação ou
a captura de perfis estudantis \cite{rigo2014aplicaccoes}.

\citeonline{bousbia2014contribution} classificaram os dados educacionais
utilizados em DM segundo as seguintes características: disponibilidade de dados,
formas de coleta, ambiente de aprendizado, descrição do nível educacional, tipo
de dado. A seguir é detalhado cada uma dessas características.

Existem três classificações para os dados de acordo com a sua disponibilidade:
Dados já disponíveis gravados ao longo dos anos na bases de dados das
instituições (e.g. notas dos estudantes, arquivos de logs dos softwares de
aprendizado), dados gerados por experimentos durante alguma pesquisa e dados
disponíveis para pesquisadores em repositórios de \textit{benchmark}, como o
PSL-Datashop\footnote{\url{https://pslcdatashop.web.cmu.edu/} Acesso em: 25 de
fev. 2019.} ou o MULCE\footnote{\url{http://repository.mulce.org} Acesso em: 25
fev. 2019.} \cite{bousbia2014contribution}.

Para as formas de coleta existem três classificações possíveis: Manual, quando
os dados são coletados por um observador humano que anota as informações para
avaliar as atividades dos participantes do processo de aprendizado. Digital,
dados gerados com o uso de \textit{hardwares} configurados para gravar as
atividades do estudante, resultam em registros em arquivos de logs, bases de
dados ou ainda arquivos de áudio ou vídeo. A terceira classificação diz respeito
às fontes de dados mistas, onde ambos os métodos são utilizados simultaneamente
\cite{bousbia2014contribution}.

No tocante ao ambiente de aprendizado existem duas classificações: Educação
tradicional, seja ensino fundamental, médio ou superior e educação baseada em
computador, i.e. sistemas de tutoria inteligente, sistemas de gerenciamento de
aprendizado, sistemas educacionais adaptativos de hipermídia, jogos
educacionais, testes e questionários \cite{romero2013data}.

Por último, a classificação segundo os tipos de dados tem as seguintes
possibilidades: Dados quantitativo ou qualitativo. Dados pessoais,
administrativos e/ou demográficos, exemplo, idade, sexo, etc. Respostas de
questionários psicológicos para medidas de satisfação, motivação, habilidade,
características cognitivas, etc., dos usuários. Interações individuais com o
sistema educacional, desde ações granulares como, cliques de mouse, até ações de
mais alto nível como, quantidade de entradas no sistema, padrão de navegação,
etc \cite{bousbia2014contribution}.

\subsection{Mineração de Dados Educacionais}

Existem diversas definições do que é Mineração de Dados Educacionais (do inglês,
\textit{Educational Data Mining}, EDM). Segundo
\citeonline{costa2012mineraccao}, a área emergente de EDM procura desenvolver ou
adaptar métodos e algoritmos de mineração existentes, de tal modo que se prestem
a compreender melhor os dados em contextos educacionais, produzidos
principalmente por estudantes e professores, considerando os ambientes nos quais
eles interagem, tais como AVAs, Sistemas Tutores Inteligentes (STIs), entre
outros.

De acordo com \citeonline{baker2009state}, EDM é definida como a área do
inquérito científico centrada em torno do desenvolvimento de métodos para
realizar descobertas ligadas aos tipos únicos de dados que provém de ambientes
educacionais, e uso desses métodos para melhor entender os estudantes e o
ambiente em que ocorre o aprendizado.

\citeonline{romero2010educational} definiram EDM como um campo que tira partido
da estatística, aprendizado de máquina e DM sobre dados educacionais com
principal objetivo de analisar esses dados  para resolver problemas
educacionais.

Muitos métodos utilizados em EDM são originalmente da área de mineração de
dados. No entanto, segundo \citeonline{baker2010data}, muitas vezes estes
métodos devem ser modificados, por se fazer necessário considerar a hierarquia
da informação. E também, existe falta de independência estatística nos tipos de
dados encontrados ao coletar informações em contextos educacionais. Logo,
diversos algoritmos e ferramentas utilizadas na área de DM não podem ser
aplicados para análise de dados educacionais sem sofrerem os devidos ajustes
\cite{baker2011mineraccao,costa2012mineraccao}.

EDM pode ser descrita como a combinação de três principais áreas (Figura
\ref{edmAreasVenn}): ciência da computação, educação e estatística. As
interseções dessas três áreas forma subáreas próximas da EDM, como sendo análise
de aprendizado (do inglês, \textit{learning analytics} (LA)), ambientes de
aprendizado baseados em computador e aprendizado de máquina
\cite{romero2013data}.

\imagem{0.85}{areas_em_correlacao_com_EDM}{\label{edmAreasVenn} Principais áreas relacionadas com EDM}{\cite{romero2013data}}

Como qualquer área interdisciplinar, a EDM utiliza métodos e aplica técnicas de
estatística, aprendizado de máquina, mineração de dados, recuperação de
informação, sistemas de recomendação, psicopedagogia, psicologia cognitiva,
psicometria, etc. A escolha de quais métodos ou técnicas devem ser utilizadas
depende da natureza do problema educacional para qual se está usando EDM
\cite{bousbia2014contribution}.

\section{Aprendizagem Supervisionada}

O campo do Aprendizado de Máquina (do inglês, \textit{Machine Learning}, ML)
fornece uma ampla área para cientistas explorarem modelos e algoritmos de
aprendizado que podem ajudar ``máquinas'' (computadores) a aprender um dado
sistema baseado em dados. Em outras palavras, o objetivo do ML é construir
sistemas inteligentes. Algoritmos de aprendizado são ferramentas de
reconhecimento de padrão. A seguir é apresentado, de uma forma geral, a
descrição de um problema de ML. Suponha que são dados um conjunto de dados e sua
respectiva resposta para um sistema. Então, o problema de ML pode ser definido
como ajustar um modelo entre eles, os dados e sua resposta, e como treinar e
validar o modelo para aprender as características do sistema através dos dados
\cite{suthaharan2016machine}.

\citeonline{russell2011artificial} definiram, em termos gerais, Aprendizagem
Supervisionada como a situação em que o agente observa alguns exemplos de pares
entrada/saída e aprende uma função que mapeia de entradas para saídas.

A tarefa de Aprendizagem Supervisionada é a seguinte:

Dados um conjunto de treinamento de \(N\) exemplos de pares entrada/saída
\[ (x_1,y_1),(x_2,y_2)\ldots(x_n,y_n), \]
onde cada \(y_j\) foi gerado por uma função desconhecida \(y = f(x)\), descobrir
uma função \(h\) que aproxime a verdadeira função \(f\)
\cite{russell2011artificial}.

Na definição acima, \(x\) e \(y\) podem ser qualquer valor, não necessariamente
numérico. A função \(h\) é uma hipótese. Aprender é procurar em um espaço de
hipóteses possíveis por uma que tenha alta performance, mesmo em exemplos não
contidos no conjunto de treinamento. Para mensurar a acurácia de uma hipótese se
utiliza um conjunto de teste, exemplos que são distintos do conjunto de
treinamento. É dito que uma hipótese generaliza bem se prediz corretamente os
valores \(y\) para exemplos novos \cite{russell2011artificial}.

Quando a saída \(y\) é uma em um conjunto finito de valores, o problema de
aprendizado é denominado classificação, e é chamado classificação booleana ou
classificação binária quando existem apenas dois valores
possíveis\cite{russell2011artificial}.
