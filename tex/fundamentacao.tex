%-------------------------------------------------------------------------------
% Este arquivo contém a sua funtamentação teórica
%-------------------------------------------------------------------------------
\chapter{Fundamentação Teórica}

A necessidade de apoiar o desenvolvimento da EAD tem feito surgir novas teorias,
métodos, abordagens de ensino e tratamento das informações, geradas nas diversas
tecnologias usadas na EAD. Serão abordadas nas seções seguintes, as principais
temáticas que serão usadas neste estudo, dando-lhe as bases teóricas necessárias
para o seu desenvolvimento e compreensão dos seus resultados futuros.

\section{Teoria da Distância Transacional}

Em 1972, Michael Grahame Moore propôs uma das primeiras teorias para a educação
a distância. Essa teoria seria posteriormente denominada de Teoria da Distância
Transacional (TDT). Ao longo de mais de 40 anos desde de seus fundamentos
iniciais, o próprio autor e outros pesquisadores trataram de atualizá-la,
principalmente em razão da evolução tecnológica e da necessidade de sua
renovação periódica. Os textos originais do autor estão nas suas obras de 1973,
1993 e 2013 \cite{moore1973transational}.

Em seus estudos, \citeonline{moore2008teoria} afirmou que, na EAD
não existe apenas uma distância física entre professores e alunos mas também uma
distância psicológica. Esta é uma teoria que descreve as relações
estudante-professor e intra-estudante e suas consequências na qualidade do
ensino à distância. Na TDT, este universo de relações pode ser estudado
baseando-se na construção dos componentes ou construtos elementares deste campo,
sendo eles, a estrutura dos programas e cursos, a interação entre alunos e
professores ou diálogo e a próprio grau de autonomia e independência do
estudante. Uma teoria educacional que afirma que a educação a distância tem a
sua própria identidade e características pedagógicas distintivas. É a primeira,
se não a única, teoria em EAD que pode ser utilizada para testar hipóteses. Como
outras teorias, pode ser usada para calcular uma heurística, utilizada para
tomada de decisões em projetos de cursos da EAD \cite{moore2008teoria}.

\citeonline{dewey1960knowing} elaboraram o conceito de transação, que, conforme
foi exposto posteriormente por \citeonline{boyd1980redefining}, denota a
interação entre o ambiente, os indivíduos e os padrões de comportamento numa
dada situação. Uma transação, em educação à distância, é a interação entre
professores e alunos que estão espacialmente separados. Como foi definido por
\citeonline{moore2008teoria}, essa separação cria padrões especiais de
comportamento que afetam tanto o ensino quanto o aprendizado. Derivado da
separação, surge um espaço psicológico e comunicacional propício a
mal-entendidos nas intervenções instrutor-aluno. A esta separação é dado o nome
de Distância Transacional (DT).

Distância Transacional é o espaço psicológico e comunicacional gerado pela
separação entre alunos e professores em um ambiente de educação à distância. O
termo distância se refere a este espaço psicológico e não a distância física
entre os participantes de um curso EAD \cite{goel2012transactional}. Esta
distância é uma função de três construtos: diálogo, estrutura do curso e
autonomia do aluno.

Faz-se necessário lembrar que, segundo \citeonline{moore2008teoria} a
distância transacional não é um valor fixo ou dicotômico, na verdade é um valor
relativo e contínuo. E é diferente para cada estudante, mesmo entre os que
compartilham o mesmo curso. Foi apontado por \citeonline{rumble1986planning} que
existe uma DT mesmo em cursos presenciais. A vista desses argumentos, podemos
dizer que a EAD é um subconjunto da educação e os estudos realizados em EAD
podem auxiliar a teoria e prática da educação tradicional. Porém, em uma
situação classificada como EAD, a distância entre os participantes - professores
e alunos - é grande o suficiente para justificar a investigação de técnicas
próprias de ensino-aprendizagem.

Os procedimentos de ensino se dividem em dois grupos, e acontece também um
terceiro grupo de variáveis que descreve o comportamento dos alunos. A DT é uma
função desses três grupos de variáveis. Essas variáveis não são tecnológicas ou
comunicacionais, na verdade, são variáveis em ensino e aprendizagem e na
interação entre ensino e aprendizagem. Em TDT, estes grupos de variáveis recebem
o nome de Diálogo, Estrutura e Autonomia do Aluno \cite{moore2008teoria}.

\subsection{Diálogo}

Diálogo foi originalmente definido por \citeonline{moore2008teoria} como
sendo interações focadas, positivas, propositais entre o professor e os alunos.
Ainda segundo Moore, o diálogo ocorre entre professores e alunos quando alguém
ensina e os demais reagem. Interações negativas ou neutras não são classificadas
como diálogo. O diálogo deve ser direcionado para o aperfeiçoamento da
compreensão por parte do aluno.

``A extensão e natureza do diálogo são determinadas pela filosofia educacional
da instituição responsável pelo projeto do curso, pelas personalidades do
professor e do aluno, pelo tema do curso e por fatores ambientais.''
\cite[p.~438]{cabau2018teoria}.

\citeonline{moore2008teoria} cita meios de comunicação como um importante
fator ambiental na EAD, no entanto, relata ser importante que outras variáveis
sejam atendidas à medida que a EAD amadurece, as variáveis destacadas por Moore
foram: projeto de curso, seleção e treinamento de instrutores e o estilo de
aprendizagem dos alunos.

É o mediador central da DT e referenciado como medida de aprendizado ao passo
que a DT seria uma medida de não-aprendizado. No entanto, já que o diálogo não
se limita apenas à interação professor-aluno, especialmente com os avanços da
EAD provendo novas formas de interações entre estudantes, diversos pesquisadores
vêm propondo a inclusão de interações/relacionamentos entre alunos no conceito
de diálogo
\cite{benson2009addressing,chen1999dimensions,huang2016understanding}.

\subsection{Estrutura do curso}

A estrutura do curso diz respeito aos elementos do projeto, bem como, divisão do
curso em unidades, objetivos, estratégias institucionais e métodos de avaliação.
Estrutura transmite a flexibilidade ou rigidez dos elementos do curso. É também
responsável pela facilitação ou não-facilitação do diálogo
\cite{moore2008teoria}.

Bem como o diálogo, a estrutura do curso é uma variável qualitativa, e a
medida da estrutura em um programa EAD é, normalmente, determinada pela natureza
dos meios de comunicação empregados, e também pela filosofia e personalidade dos
professores, pelas personalidades dos alunos e pelas restrições impostas pelas
instituições educacionais \cite{moore2008teoria}.

Em cursos gravados em fitas, discos, ou mesmo cursos televisionados a estrutura
é rígida e o diálogo não existe, pois não é possível reorganizar o conteúdo para
levar em consideração as interações de um aluno. Em contrapartida cursos por
teleconferências, permitem ampla variedade de respostas alternativas do
instrutor às perguntas dos participantes. Um curso altamente estruturado não
possibilita o diálogo professor-aluno, consequentemente a DT entre alunos e
professores aumenta. No entanto, o contrário não pode ser generalizado. ``\ldots
a extensão do diálogo e a flexibilidade da estrutura variam de programa para
programa. É esta variação que dá a um programa maior ou menor distância
transacional que outro'' \cite{moore2008teoria}.

Em um programa com pequena DT os alunos recebem instruções e orientações por
meio do diálogo com o instrutor, neste caso é possível ter uma estrutura aberta,
que dê respaldo para tais interações. Em programas com maior DT é necessário uma
estrutura robusta, materiais didáticos que forneçam todas as orientações,
instruções e aconselhamentos que o instrutor puder prever, mas sem a
possibilidade de alterações por meio de diálogo aluno-professor
\cite{moore2008teoria}.

Temos então que, em programas mais distantes, os alunos precisam se
responsabilizar em escolher quais atividades e avaliações serão feitas e quando
serão feitas. Mesmo que o curso seja bem estruturado, o estudante, na falta de
diálogo, decidirá quais atividades serão realizadas, quando, e qual a
importância de cada uma. Sendo assim, quanto maior a DT mais é exigido uma
autonomia do aluno \cite{moore2008teoria}.


\subsection{Autonomia do aluno}

No período do lançamento da TDT, década de 1970, ela representava a fusão de
duas tradições pedagógicas presentes nos anos 1960 que pareciam contraditórias.
Uma, a tradição humanística, que valorizava o diálogo aberto, não-estruturado e
interpessoal, tanto na educação quanto no aconselhamento. A outra, a tradição
behaviorista que valorizava o projeto sistemático da instrução, baseado em
objetivos comportamentais com o máximo de controle do processo de aprendizagem
por parte do professor. No início dos anos 1970, a EAD era dominada pela
tradição behaviorista. Tanto que, o título do primeiro trabalho sobre a TDT de
\citeonline{moore1972learner} foi: ``A autonomia do aluno --- a segunda dimensão
da aprendizagem independente''. Neste trabalho Moore afirmou que: ``educadores
por correspondência universitários limitavam o potencial do seu método ao
negligenciar a habilidade dos alunos em compartilharem a responsabilidade por
seus próprios processos de aprendizagem'' \cite{moore2008teoria}.

O termo ``autonomia do aluno'' foi escolhido para descrever os padrões de
comportamento de alunos que usavam materiais didáticos e programas de ensino
para atingir seus próprios objetivos, à sua maneira e sob seu próprio controle
\cite{moore2008teoria}.

\apudonline{boyd1966psychological}{moore2008teoria} criou uma descrição para o
comportamento ideal de uma pessoa emocionalmente independente de um instrutor,
uma pessoa que pode abordar assuntos diretamente sem ter um adulto participando
de um conjunto de papéis de mediação entre o aluno e a matéria de estudo.

Autonomia do aluno  se refere a capacidade de se auto-direcionar.
\citeonline{moore2008teoria} definiu o estudante autônomo ideal como ``a
pessoa emocionalmente independente de um professor'' e quem ``tem capacidade de
abordar o assunto estudado diretamente sem a ajuda de um instrutor''. Diferente
da estrutura do curso e do diálogo, é um fator que depende apenas do aluno. Um
aprendiz pouco autônomo pode precisar de um direcionamento maior e uma estrutura
mais rígida \cite{huang2016understanding}.

\section{Evasão de alunos na EAD}

O censo mais recente, no período da escrita deste trabalho, realizado pela
Associação Brasileira de Educação a Distância (ABED), juntado dados de 2016,
consultou 340 instituições em todo o país, formadoras e fornecedoras de de
produtos e serviços para EAD. 36 instituições são formadoras e fornecedoras e 28
são exclusivamente fornecedoras, para fins de análise o censo considerou o total
de formadoras como 312 instituições. Foram ouvidas 54 instituições públicas
federais, 26 instituições públicas estaduais, 6 instituições públicas
municipais, 32 instituições do Sistema Nacional de Aprendizagem (SNA): Senai,
Sesi, Senac, Senat, Sebrae etc., 106 instituições privadas sem fins lucrativos,
14 órgãos públicos ou instituições do governo e 10 organizações não
governamentais (ONGs). Neste censo foram contabilizados mais de 550 mil alunos
em cursos regulamentados totalmente a distância, mais de 200 mil em cursos
regulamentados semipresenciais,  quase 2 milhões em cursos livres não
corporativos e mais de 1,25 milhões em cursos livres corporativos
\cite{abed2016ead}.

De acordo com o Censo EAD.BR 2016 as taxas de evasão informadas pelos
respondentes recaíram principalmente entre 11\% e 25\%. O censo também revelou
que entre os respondentes, as taxas de evasão dos cursos semipresenciais tem
taxa de evasão menor que cursos totalmente a distância. A Tabela
\ref{tableEvasionTax1} compara os índices dos 4 últimos censos realizados pela
ABED
\cite{abed2013ead,abed2014ead,abed2015ead,abed2016ead}

\begin{table}[!htb]
  \centering
  \caption{\label{tableEvasionTax1} Taxas de evasão ao longo dos anos segundo o censo realizado pela ABED}
  \begin{tabular}{@{}lllll@{}}
    \toprule
    \multicolumn{1}{c}{\multirow{2}{*}{\textbf{\begin{tabular}[c]{@{}c@{}}Taxas de evasão \\ declaradas\end{tabular}}}} & \multicolumn{4}{c}{\textbf{\begin{tabular}[c]{@{}c@{}}Percentuais de instituições\\ declarantes, por faixa\end{tabular}}} \\ \cmidrule(l){2-5}
    \multicolumn{1}{c}{} & \textbf{2013} & \textbf{2014} & \textbf{2015} & \textbf{2016} \\ \midrule
    Até 25\% & 65\% & 50\% & 53\% & 58\% \\
    Entre 26 e 50\% & 24\% & 38\% & 40\% & 19\% \\
    Acima de 50\% & 2\% & 2\% & 7\% & 1\% \\
    Não declararam & 9\% & 10\% & - & 22\% \\ \bottomrule
  \end{tabular}
  \legend{\textbf{Fonte:} \citeonline{abed2013ead,abed2014ead,abed2015ead,abed2016ead}}
\end{table}

Entre os motivos para evasão investigados e declarados no censo, questões
financeiras e falta de tempo foram os citados como os que geram maior evasão.
Houve uma parcela considerável de respondentes que acredita que a evasão não é
um problema em cursos totalmente a distância pois os participantes podem sempre
retornar.

Em cursos livres o motivo mais citado foi a falta de tempo, e também grande
parte dos respondentes acredita que os alunos de cursos livres não corporativos
sempre podem retornar.

O Censo EAD.BR 2016 apontou que cursos presenciais, semipresenciais e
corporativos possuem mecanismos que vão além do conteúdo e da interação online
com professores para manter seus alunos engajados. Já os cursos totalmente a
distância e cursos livres não corporativos dependem apenas da experiência do
aluno com o conteúdo e com seus professores e tutores.

A Tabela \ref{tableEvasionTax2} apresenta dados dos indicadores da evasão
segundo o Mapa do Ensino Superior no Brasil Edições 2015 e 2016, que foram
publicados pelo Sindicato das Empresas Mantenedoras do Ensino Superior (SEMESP)
feito com base nos dados do INEP dos anos 2013 e 2014.

\begin{table}[!htb]
  \centering
  \caption{\label{tableEvasionTax2} Taxas de evasão em cursos superiores presenciais e a distância}
  \begin{tabular}{@{}ccccc@{}}
    \toprule
    \multirow{2}{*}{\textbf{Ano}} & \multicolumn{2}{c}{\textbf{Cursos presenciais}} & \multicolumn{2}{c}{\textbf{Cursos a Distância}} \\ \cmidrule(l){2-5}
    & \textbf{IES públicas} & \textbf{IES privadas} & \textbf{IES públicas} & \textbf{IES privadas} \\ \midrule
    2013 & 65\% & 50\% & 53\% & 58\% \\
    2014 & 24\% & 38\% & 40\% & 19\% \\ \bottomrule
  \end{tabular}
  \legend{\textbf{Fonte:} \citeonline{semesp2015mapa,semesp2016mapa}}
\end{table}

Segundo o trabalho de \apudonline{ribeiro2005projeto}{paz2017identificando} a
evasão em instituições de ensino superior (IES) é um tema complexo na gestão
universitária no Brasil. E, segundo
\apudonline{costa2012minerando}{paz2017identificando}, um grave problemas das
universidades brasileiras é o aumento das taxas de evasão escolar.

\citeonline{manhaes2011previsao} identificaram que a descoberta precoce de
grupos de estudantes com risco de evasão é condição importante para reduzir tal
problema pois possibilita proporcionar algum tipo de atendimento personalizado
para a situação de cada aluno. Ainda segundo \citeonline{manhaes2011previsao},
os processos de identificação desses grupos à época eram manuais e sujeitos a
falhas e dependiam primordialmente da experiência do docente.

Motivado pelos problemas elencados acima este trabalho se baseia em técnicas de
mineração de dados e aprendizado de máquina (AM) para tentar categorizar,
previamente, o comportamento de alunos da EAD na Univasf e mitigar os problemas
da evasão escolar.

\section{Relação entre a distância transacional e a evasão em cursos a distância}

Pela sua definição, a distância transacional é um dos fatores que pode gerar
maior dificuldade no engajamento significativo e na comunicação do estudante no
ambiente de aprendizagem \cite{goel2012transactional}. Além de
\citeonline{moore2008teoria}, outros autores afirmaram que quanto maior for a
distância transacional, maior a possibilidade de ocorrência de problemas como
atritos, insatisfações e abandono de cursos
\cite{zhang2003transactional,steinman2007educational,horzum2011developing,
mbwesa2014transactional,paul2015revisiting}.

\citeonline{zhang2003transactional}, em sua tese de doutorado, demonstrou a
existência de uma correlação negativa entre a distância transacional e o
envolvimento dos alunos com a sua aprendizagem, assim como com a sensação de
satisfação e a intenção do aluno em persistir no seu curso on-line.

Os elementos da distância transacional são preditores do engajamento e da
satisfação dos alunos em cursos on-line, o que, de certa forma, são elementos
relacionados com a decisão de permanecer ou abandonar o curso, segundo a revisão
de \citeonline{paul2015revisiting} do trabalho de
\citeonline{zhang2003transactional}.

Para \citeonline{steinman2007educational}, as percepções dos alunos de cursos
on-line podem ser negativas se eles experimentam grande distância transacional
com o instrutor e com outros alunos, podendo ainda influenciar sua decisão de
permanecer ou abandonar o curso. Uma vez que a distância transacional afeta a
satisfação e retenção dos alunos, esse conceito é visto como um importante
tópico de discussão sobre evasão em cursos on-line.

A obtenção dos construtos da distância transacional pode refletir uma condição
ou um estado de um curso no tempo de sua execução, permitindo, por exemplo, que
professores e tutores notem um distanciamento exagerado de determinados alunos e
consigam intervir no sentido de prevenir ou reverter situações de evasão de
alunos do curso \cite{horzum2011developing}.

Em sua pesquisa, \citeonline{mbwesa2014transactional}, procurou estabelecer os
construtos da distância transacional como estimadores da satisfação dos alunos
em um curso a distância. De acordo com os resultados do estudo, as distâncias
transacionais verificadas entre aluno-aluno, aluno-professor e aluno-conteúdo
foram preditores importantes da satisfação percebida dos alunos com os cursos
por EAD.